% Search for all the places that say "PUT SOMETHING HERE".

\documentclass[11pt]{article}
\usepackage{amsthm}% http://ctan.org/pkg/amsthm
\usepackage{etoolbox}% http://ctan.org/pkg/etoolbox
\usepackage{amssymb}% http://ctan.org/pkg/amssymb
\usepackage{pifont}% http://ctan.org/pkg/pifont
\usepackage{amsmath}
\usepackage{graphicx}
\usepackage{xypic}
\usepackage{caption}
\usepackage{hyperref} 

\newcommand{\cmark}{\ding{51}}%
\newcommand{\xmark}{\ding{55}}%
\newcommand\tab[1][1cm]{\hspace*{#1}}

\title{CS70--Fall 2011 --- Solutions to Homework 2}

\begin{document}
	\maketitle
	
	\begin{enumerate}
		\item Practice Proving Propositions
		
		\begin{enumerate}
			\item \textbf{Claim:} For all natural numbers $n$, if $n$ is even then $n^2+2011$ is odd.
			
			\textbf{Proof:} \textit{by Induction} \\
			Let $$ \tab P(n) = n^2+2011 \tab \forall n \in N$$
			
			\textbf{Base Case:} 
			
			$$P(0) = 0^2 + 2011 = 2011$$
			
			which is odd. So the base case is true.\\
			
			\textbf{Inductive Step:}
			
			Assume $P(n)$ is true. Taking $P(n+2)$
			
			$$P(n+2) = (n+2)^2+2011$$  
			$$P(n+2) = n^2 + 4n + 2015$$ 
						
			Take $K(q) = q^2 + 4q + 2015$. There is only one possibility 
			\begin{enumerate} 
							
				\item $q$ is even. If $q$ is even, $q^2$ is also even. Moreover $4q$ becomes even. So we have
				$$K(q) = Even + Even + Odd $$ 
				$$K(q) = Even + Odd $$ 
				$$K(q) = Odd \text{ \cmark}$$ 
							
			\end{enumerate}
			
			So, $P(n+2) = n^2 + 4n + 2015$ is odd. So, $P(n) \rightarrow P(n+2)$ is true. The claim is correct. \cmark \\
		
			\item \textbf{Claim:} For all natural numbers $n$, $n^2+5n+1$ is odd.
			
			\textbf{Proof:} \textit{by Induction} \\
			Let $$ \tab P(n) = n^2+5n+1 \tab \forall n \in N$$
			
			\textbf{Base Case:} 
			
			$$P(0) = 0^2 + 5(0) + 1 = 1$$
			
			which is odd. So the base case is true.\\
			
			\textbf{Inductive Step:}
			
			Assume $P(n)$ is true. Taking $P(n+1)$
			
			$$P(n+1) = (n+1)^2 + 5(n+1) + 1$$  
			$$P(n+1) = n^2 + 7n + 7$$ 
			
			Take $K(q) = q^2 + 7q + 7$. There are two possibilities 
			\begin{enumerate}
				\item $q$ is odd. If $q$ is odd, $q^2$ is also odd. Morever $7q$ becomes odd. So we have
				$$K(q) = Odd + Odd + Odd $$ 
				\textbf{Axiom}: Adding an Odd number with an Odd number gives an even number.
				$$K(q) = Even + Odd $$ 
				\textbf{Axiom}: Adding an Even number with an Odd number gives an Odd number.
				$$K(q) = Odd \text{ \cmark}$$ 
				
				\item $q$ is even. If $q$ is even, $q^2$ is also even. Moreover $7q$ becomes even. So we have
				$$K(q) = Even + Even + Odd $$ 
				$$K(q) = Even + Odd $$ 
				$$K(q) = Odd \text{ \cmark}$$ 
				
			\end{enumerate}
			
			
			So, $P(n+1) = n^2 + 7n + 7$ is odd. So, $P(n) \rightarrow P(n+1)$ is true. The claim is correct. \cmark \\
			
			
			\item \textbf{Claim:} For all real numbers $a,b$, if $a+b\ge 2011$ then $a > 1005$ or $b > 1005$.
			
			\textbf{Proof:} \textit{Direct Proof}
			
			Let $$P(a, b) = (a+b\ge 2011)$$ and $$Q(a, b) = (a > 1005 \text{ or } b > 1005)$$
			
			We have to prove that $P(a, b) \rightarrow Q(a, b)$ is true. Just take $a$ to be 1005, therefore for the implication to be true $b$ has to be greater than 1005, because otherwise $Q(a, b)$ will be false, and so will the implication. 
			
			$$a + b \ge 2011$$
			$$1005 + b \ge 2011$$
			$$b \ge 1006 $$ 
			$$b > 1005 $$ which implies that $Q(a, b)$ is true. \cmark
			
			\item \textbf{Claim:} For all real numbers $r$, if $r$ is irrational then $r/4$ is irrational.
			
			\textbf{Proof:}  \textit{by Contraposition}
			
			Let $P(r)$ be the statement that $r$ is irrational, and $Q(r)$ be the statement that $r/4$ is irrational. Thus, we have to prove $$P(r) \rightarrow Q(r)$$ By Contraposition, $$\lnot Q(r) \rightarrow \lnot P(r)$$.
			
			Let $X(r)$ be the proposition that $r/4$ is rational, and $Y(r)$ be the proposition that $r$ is rational. So we have to prove, $$X(r) \rightarrow Y(r)$$ 
			
			Given that $r/4$ is a rational number, we have 
			\[ \left(\frac{r}{4}\right) = \left(\frac{p}{q}\right) \]
			
			where $p, q \in Z$
			
			\[ r = \left(\frac{4p}{q}\right) \]
 			
 		    \textbf{Axiom}: \textit{Given p is an integer, np is also integer where n $\in$ Z} 
 		    
 		    Thus, 
 			
 			\[ r = \left(\frac{m}{q}\right) \]
 			
 			where $m = 4 \cdot p$, is rational. The claim is correct. \cmark
 			
			\item \textbf{Claim:} For all natural numbers $n$, $10n^2 > n!$.
			
			\textbf{Proof:} \textit{by Counterexample}
			
			Let $P(n)$ be the proposition $10n^2 > n!$ where $n \in N$. We could simply iterate through a couple of elements in $N$ until we find a $n \in N$ for which $P(n)$ doesn't hold. \\
			

		\begin{enumerate}
		
			\item	$P(1) = (10 > 1)$ \cmark 
			\item	$P(2) = (40 > 2)$ \cmark 
			\item	$P(3) = (90 > 6)$ \cmark 
			\item	$P(4) = (160 > 24)$ \cmark 
			\item	$P(5) = (250 > 120)$ \cmark 
			\item	$P(6) = (360 > 720)$ \xmark 
		
		\end{enumerate}	
			
			At $n = 6$, we have a counterexample to the claim $P(n)$. Thus the claim is incorrect. \xmark
			
		\end{enumerate}
		
		\newpage
		\item Interesting Induction
		
		\begin{enumerate}
			\item
			For $n \in N$ with $n\ge 2$, define $s_n$ by
			\[ s_n = \left(1 - \frac{1}{2}\right) \times \left(1 - \frac{1}{3}\right) \times \dots
			\times \left(1 - \frac{1}{n}\right). \]
			
			\textbf{Claim:} $s_n = 1/n$ for every natural number $n\ge 2$.
			
			\textbf{Proof:}  \textit{by Induction}
			
			Given \[ s_n = \left(1 - \frac{1}{2}\right) \times \left(1 - \frac{1}{3}\right) \times \dots
			\times \left(1 - \frac{1}{n}\right). \]
			
			\textbf{Base Case:} 
						
			\[ s_2 =  \left(1 - \frac{1}{2}\right) = \left(\frac{1}{2}\right) \text{ \cmark}\]
				
			\textbf{Inductive Step:}
			
			Assume $s_n$ to be true. 
			
			\[ s_{n+1} = \left(1 - \frac{1}{2}\right) \times \left(1 - \frac{1}{3}\right) \times \dots
			\times \left(1 - \frac{1}{n}\right) \times \left(1 - \frac{1}{n+1}\right). \]
			
			\[ s_{n+1} = \left(\frac{1}{2}\right) \times \left(\frac{2}{3}\right) \times \dots
			\times \left(\frac{n - 1}{n}\right) \times \left(\frac{n}{n+1}\right). \]
			
			\[ s_{n+1} = \left(\frac{1 \cdot 2 \cdots (n-1) \cdot n}{2 \cdot 3 \cdots n \cdot (n+1)}\right). \]			
			
			\[ s_{n+1} = \left(\frac{n!}{(n+1)!}\right)  
			\]
			
			\[ s_{n+1} = \left(\frac{1}{n} \right) \text{\cmark}
			\]
			
			The claim is correct. \cmark.
			
			\item
			Let $a_n = 3^{n+2} + 4^{2n+1}$.
			\textbf{Claim:} 13 divides $a_n$ for every $n \in N$.
			
			\textbf{Proof:} \textit{by Induction}
			
			Given $$a_n = 3^{n+2} + 4^{2n+1}$$
			
			\textbf{Base Case}:
			$$a_0 = 3^{2} + 4 = 13$$
			$$13 \vert a_0 = 13 \vert 13 \text{ \cmark}$$
			
			\textbf{Inductive Step}:
			Assume $a_n$ to be true. 
			
			$$a_{n+1} = 3^{n+3} + 4^{2n+3}$$
			
			Using the Hint
			
			$$a_{n+1} - 3a_{n} = (3^{n+3} + 4^{2n+3}) - (3^{n+2} + 4^{2n+1})$$
			\\ After solving algebraically 
			$$a_{n+1} - 3a_{n} = 52 \cdot 4^{2n}$$			
			
			$$13 \vert (a_{n+1} - 3a_{n}) = 13 \vert 52 \cdot 4^{2n} \text{ \cmark}$$
			
			This means that 
			$$(13 \vert a_{n+1}) - (13 \vert 3a_{n})$$ is true. Which tells us that $13 \vert a_{n+1}$ is true. \cmark
			
			
		\end{enumerate}
		
		\newpage
		\item Proofs, Perhaps
		
		\begin{enumerate}
			
			\item This claim is false, and the proof is incorrect. The problem lies in the inductive step. The inductive hypothesis is $$P(k) = (k^2 < k)$$ In Mathematical Induction, you assume $P(k)$ is true and prove that $P(k+1)$ is true, and therefore prove that $$P(k) \rightarrow P(k+1)$$ is true $\forall k \in D$ where $D$ is some domain. However in this proof, you prove something(the inductive hypothesis) that you already had assumed to be true, which is incorrect. 
			\item This claim is false, and the proof is incorrect. In the Inductive step, you have $$7^{k+1} = (7^k \cdot 7^k) / 7^{k-1}$$ However, when k = 0
			 
			$$7^1 = (7^0 \cdot 7^0) / 7^{-1}$$ 
			
			The claim only holds for $k \in N$, and $-1 \notin N$. Therefore, you can't substitute $1$ in its place, and the proof breaks. 
			 
			\item This claim is correct, and so is the proof. 
			\item This claim is false, and the proof is incorrect. The problem is in the claim that 
			$$ max\text[(a - 1),(b - 1)] = n$$ because this doesn't hold hold for $a = b = 0$ because $-1 \notin N$. 
		\end{enumerate}
		
		\newpage
		\item Take the Tokens
		
		\textbf{Claim:} For all natural numbers \(k\), if the pile starts with \(4k +
		1\) tokens, then Thuc has a winning strategy.
		
		\textbf{Proof:}  \textit{by Induction}
		
		Let $P(k) = 4k + 1 \tab \forall k \in N$
		
		\textbf{Base Case:} 
		
		$$P(0) = 4(0) + 1 = 1$$
		
		If there's only 1 token, then Tamara will end up picking it firstly, and thus Thuc wins. \\
		
		\textbf{Inductive Step:}
		
		Assume $P(k)$ is true. Taking $P(k+1)$
		
		$$P(k+1) = 4(k+1) + 1$$  
		$$P(k+1) = (4k + 1) + 4$$ 
		$$P(k+1) = P(k) + 4$$
		
		Tamara has to go first, and can pick up either 1, 2 or 3 tiles. Thus, Thuc can afterwards pick 
		$$4 - \textit{No of tiles Tamara picked}$$ to get the number of tiles in the form of $P(k) = 4k + 1$, under which condition Thuc knows he will win. \cmark
		
		
		\newpage
		\item Rigorous Recursion
		
		\textbf{Claim:} For all inputs $n\in\mathbf{N}$, the value returned by the program is $G(n)=3^n-2^n$.
		
		\textbf{Proof:} \textit{by Induction}
		
		Given $G(n) = 3^n - 2^n$
		
		\textbf{Base Case}:
			\begin{enumerate}
				\item \textit{if (n = 0)}
					$$G(0) = 3^0 - 2^0 = 0 \text{ \cmark}$$  
				\item \textit{if (n = 1)}
					$$G(1) = 3^1 - 2^1 = 1 \text{ \cmark}$$ 
			
			\end{enumerate}
			
		
			\textbf{Inductive Step}:
			Assume G(k) is true $\forall 0 \le k \le i$. Solving for $G(k+1)$ 
					$$5G(k + 1 - 1) - 6G(k + 1 - 2) = 5G(k) - 6G(k - 1)$$
					$$= 5(3^{k} - 2^{k}) - 6(3^{k-1} - 2^{k-1})$$ 
					After algrebic manipulation
					$$ = 3^{k+1} - 2^{k+1} \text{ \cmark } $$
					
			
		
		
		
		\newpage
		\item Coloring Countries (Some of the content here including pictures has been taken from \url{http://mathonline.wikidot.com/6-colour-theorem-for-planar-graphs})
		
		\begin{enumerate}
			
			\item
			\textbf{Theorem:}
			
			Every possible map can be colored with six colors, in such a way that no
			two neighboring countries have the same color.
			
			\textbf{Proof:} \textit{by Induction} \\
			Let $P(n)$ be the statement that the vertices in a planar graph can be coloured with 6 colors. 
			
			\textbf{Base Case:} \\
			For $1 \le n \le 6$, $P(n)$ is correct because you can give each and every vertex a different color than its neighbor. 
			
			\textbf{Inductive Step:} \\
			Assume that $P(n)$ is true, which means that if we have a planar graph with $n$ vertices, we can obtain a good coloring. We want to prove that if we have $n+1$ vertices, the coloring is still possible. 
			
			Now, assume $G$ is a planar graph with $n + 1$ vertices. The lemma tells us that the degree of the vertex $ \le 5$. Therefore, we would have a graph as shown in the figure below.
			
			\includegraphics[width=\textwidth,height=\textheight,keepaspectratio]{a.png}
			
			If we remove the vertex $v$, the number of vertices become $n$, for which we know that good coloring is possible, as in the following figure . 

			\includegraphics[width=\textwidth,height=\textheight,keepaspectratio]{b.png}

			Adding the vertex $v$ back, the figure becomes

			\includegraphics[width=\textwidth,height=\textheight,keepaspectratio]{c.png}
			
		Thus, a good coloring is possible. So $P(n+1)$ is true. \cmark
			
		\end{enumerate}
		
	\end{enumerate}
\end{document}
