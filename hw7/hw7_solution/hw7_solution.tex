\documentclass[11pt]{article}
\usepackage{amsthm}% http://ctan.org/pkg/amsthm
\usepackage{etoolbox}% http://ctan.org/pkg/etoolbox
\usepackage{amssymb}% http://ctan.org/pkg/amssymb
\usepackage{pifont}% http://ctan.org/pkg/pifont
\usepackage{amsmath}
\usepackage{graphicx}
\usepackage{xypic}
\usepackage{caption}
\usepackage{hyperref} 
\usepackage{listings}
\usepackage{color}

\definecolor{dkgreen}{rgb}{0,0.6,0}
\definecolor{gray}{rgb}{0.5,0.5,0.5}
\definecolor{mauve}{rgb}{0.58,0,0.82}

\lstset{frame=tb,
	language=c++,
	aboveskip=3mm,
	belowskip=3mm,
	showstringspaces=false,
	columns=flexible,
	basicstyle={\small\ttfamily},
	numbers=none,
	numberstyle=\tiny\color{gray},
	keywordstyle=\color{blue},
	commentstyle=\color{dkgreen},
	stringstyle=\color{mauve},
	breaklines=true,
	breakatwhitespace=true,
	tabsize=3
}

\hypersetup{
	colorlinks   = true,
	citecolor    = gray
}

\newcommand{\cmark}{\ding{51}}%
\newcommand{\xmark}{\ding{55}}%
\newcommand*{\permcomb}[4][0mu]{{{}^{#3}\mkern#1#2_{#4}}}
\newcommand*{\perm}[1][-3mu]{\permcomb[#1]{P}}
\newcommand*{\comb}[1][-1mu]{\permcomb[#1]{C}}

%%% - The heading
\title{CS70--Fall 2011 --- Solutions to Homework 7}
\pagestyle{myheadings}

%%% - Don't show the Date
\date{}
%%% - Don't show the Page numbers
\pagenumbering{gobble}

\begin{document}
	\maketitle

	%%% Enumeration for the Questions
	\begin{enumerate}
		
		%%% Q1
		\item \textbf{Sample Space and Events}
			
			%%% Parts of Q1
			\begin{enumerate}
				\item $\Omega$ has $16$ total outcomes. \\ 
				HHHH, HTHH, THHH, TTHH,	HHHT, HTHT, THHT, TTHT, \\
				HHTH, HTTH, THTH, TTTH, HHTT, HTTT, THTT, TTTT
				\item A has $8$ possible outcomes. \\
				HHHH, HTHH, HHHT, HTHT, HHTH, HTTH, HHTT, HTTT
				\item B has $8$ possible outcomes. \\
				HHHH, HTHH, THHH, TTHH, HHHT, HTHT, THHT, TTHT
				\item C has $4$ possible outcomes. \\
				HHHH, HTHH, HHHT, HTHT
				\item D has $12$ possible outcomes. \\
				HHHH, HTHH, THHH, TTHH, HHHT, HTHT 
				THHT, TTHT, HHTH, HTTH, HHTT, HTTT
				\item Like you can see from the outcomes, A and B are not disjoint. The relationships are \\
					\begin{center}
						C = A \textbf{AND} B \\
						D = A \textbf{OR} B
					\end{center}
				\item I leave the details of this up to you. 
					\begin{center}
						$$\vert \Omega \vert = 2^n$$
						$$\vert A \vert = 2^{n-1}$$
						$$\vert B \vert = 2^{n-1}$$
						$$\vert C \vert = 2^{n-2}$$
						$$\vert D \vert = 2^n - 2^{n-2}$$
					\end{center} 
				\item $\frac{2}{3}$. If you see carefully, this problem is essentially the Monty Hall problem in disguise. \\
				An outcome consists of us selecting a coin, and then choosing one of its sides. Since there are $3$ coins
				each with $2$ sides, our sample space $\Omega$ has $6$ elements
				$$\Omega = \left\{(HH,H1),(HH,H2),(HT,H),(HT,T),(TT,T1),(TT,T2)\right\}$$
				where $(HH,H1)$ refers to the outcome of drawing the coin with two heads and of looking at the first
				side. Similarly, $(HH,H2)$ refers to the outcome of choosing the coin with two heads and of looking at
				the second side, $(HT,T)$ refers to the outcome of choosing the coin with both heads and tails and of
				looking at the tails side, etc. Let A be the event that we choose the HH coin, and let B be the event
				that we see a heads when we put the coin down on the table. We wish to compute $Pr[A \vert B]$. Since
				$A = {(HH,H1),(HH,H2)}$, $B = {(HH,H1),(HH,H2),(HT,H)}$, and A $\cap$ B = A, we see
				$$Pr[A \vert B] = \frac{Pr[A]}{Pr[B]} = \frac{2}{3} $$

				
			\end{enumerate}
		
		%%% Q2
		\item \textbf{Monty Hall Revisited} \\
		\textbf{Solution}: $\frac{2}{3}$. Consider three doors. Each of these doors has a probability of $\frac{1}{3}$ of having the prize, because they are independent from one another. Lets call them A, B and C. So, 
					$$P(A) = P(B) = P(C) = \frac{1}{3}$$
		Suppose you select one of these doors, say A. As mentioned, the probability that A has the prize is mere $\frac{1}{3}$. Therefore, the probability that the prize is in B or C(not in A is $\frac{2}{3}$. Now the host opens either B or C, if it doesnot have the prize. Suppose he opens B. Now because B doesnot have the prize, the 
							$$P(B) = 0$$
		but we already know that 
							$$P(A) = \frac{1}{3}$$ 
		because total probability has to be 1. Thus, 
							$$P(C) = \frac{2}{3}$$
		\textbf{Note: } For more explanation, please refer to \href{https://www.youtube.com/watch?v=4Lb-6rxZxx0}{This} and \href{http://www.archive.org/download/MIT6.042JF10/MIT6_042JF10_lec18_300k.mp4}{This} for more elaborate explanation. 
		
		%%% Q3
		\item \textbf{Rolling Dice}
			
			The details are left to you. These are pretty straight forward problems btw. 
			%%% Parts of Q3
			\begin{enumerate}
				\item $\frac{1}{6}$
				\item $\frac{1}{3}$
				\item $\frac{11}{36}$
				\item $\frac{1}{3}$
			\end{enumerate}
		
		
	\end{enumerate}
	
\end{document}