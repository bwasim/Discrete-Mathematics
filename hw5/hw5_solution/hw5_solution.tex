\documentclass[11pt]{article}
\usepackage{amsthm}% http://ctan.org/pkg/amsthm
\usepackage{etoolbox}% http://ctan.org/pkg/etoolbox
\usepackage{amssymb}% http://ctan.org/pkg/amssymb
\usepackage{pifont}% http://ctan.org/pkg/pifont
\usepackage{amsmath}
\usepackage{graphicx}
\usepackage{xypic}
\usepackage{caption}
\usepackage{hyperref} 
\usepackage{listings}
\usepackage{color}

\definecolor{dkgreen}{rgb}{0,0.6,0}
\definecolor{gray}{rgb}{0.5,0.5,0.5}
\definecolor{mauve}{rgb}{0.58,0,0.82}

\lstset{frame=tb,
	language=c++,
	aboveskip=3mm,
	belowskip=3mm,
	showstringspaces=false,
	columns=flexible,
	basicstyle={\small\ttfamily},
	numbers=none,
	numberstyle=\tiny\color{gray},
	keywordstyle=\color{blue},
	commentstyle=\color{dkgreen},
	stringstyle=\color{mauve},
	breaklines=true,
	breakatwhitespace=true,
	tabsize=3
}

\newcommand{\cmark}{\ding{51}}%
\newcommand{\xmark}{\ding{55}}%


%%% - The heading
\title{CS70--Fall 2011 --- Solutions to Homework 5}
\pagestyle{myheadings}

%%% - Don't show the Date
\date{}
%%% - Don't show the Page numbers
\pagenumbering{gobble}

\begin{document}
	\maketitle
	
	%%% begin enumeration for the Questions. 
	\begin{enumerate}
		
		%%% Q1
		\item \textbf{Euclid's argument} \\
		%%% Begin Solution to Q1
		\textbf{Solution: } \\ Although Euclid's argument is valid, the given proof is not correct. In order to get a contradiction, Euclid assumes that there are only $k$ primes $p1 p2 \cdots pk$. This false assumption helps him to show that $p1 p2 \cdots pk + 1$ is a prime as well. However, there might be prime numbers beyond the first $k$. Lets take an example with $k = 6$. Under this condition, we will have 
		
		$$2 \cdot 3 \cdot 5 \cdot 7 \cdot 11 \cdot 13 + 1 = 30031$$
		
		Under the assumption that $\{2, 3, 5, 7, 11, 13\}$ are the only prime numbers, the number $30031$ is also a prime, because none of the numbers in the list divides $30031$. However, the assumption that $\{2, 3, 5, 7, 11, 13\}$ are the only prime numbers is incorrect. In fact, there is a prime number $59$ such that 
					$$59 \cdot 509 = 30031$$ 
		
		Because of the faulty assumption that there are only $k$ prime numbers, the proof is incorrect. 
		%%% End Solution to Q1
		
		%%% Q2
		\item \textbf{GCD}
		%%% Begin Solution to Q2
		%%% As the problem has two parts, we'll do an enumeration 
			\begin{enumerate}
				%%% Q2 - A
				\item Use extended euclid’s algorithm to find some pair of integers j, k such that $52 j + 15k = 3$. \\
				%%% Begin Solution to Q2 - A
				\textbf{Solution: } \\
				1- Applying the algorithm, 
							$$52 - 3(15) = 7$$
							$$15 - 2(7) = 1$$
							$$7 - 7(1) = 0$$
				
				2- Solving, 
							$$1 = 15 - 2(7)$$
							$$= 15 - 2(52 - 3(15))$$
							$$= 7 \cdot 15+(-2) \cdot 52$$
				
				Multiplying both sides by $3$ gives 
				$$3 = 3 \cdot 7 \cdot 15 + 3(-2)52 = 21 \cdot 15+(-6)52$$
				
				So, $j = -6$ and $k = 21$
				%%% End Solution to Q2 - A
				
				%%% Q2 - B
				\item If $gcd(m,x) > 1$, how many distinct elements are there? \\
				%%% Begin Solution to Q2 - B
				\textbf{Solution: } \\ We already know that if $gcd(m, x) = 1$, then the set $\{ mod(ax, m) : a \in \{0, 1, \cdots , m - 1\}\}$ has $m$ elements. We can use this proof to find the number of elements when $gcd(m, x) > 1$. We have, 
							$$ax \equiv y (\text{mod  } m)$$
				and, 
							$$ax - y = km$$
				We want to find numbers $(x', m')$ such that $gcd(x', m') = 1$. Then the number of distinct elements would be $m'$. This is possible if we divide both sides by $gcd(m, x)$, because this will remove the common factor from both sides. Let $d = gcd(m, x)$. We have,
					$$a \frac{x}{d} - \frac{y}{d} = k \frac{m}{d}  $$
				Note that $gcd(\frac{x}{d}, \frac{m}{d}) = 1$. So the total number of elements when $gcd(m, x) > 1$ is 		
				
				$$Number of elements = \frac{m}{d} = \frac{m}{gcd(m, x)} $$
				
				%%% End Solution to Q2 - B				
				
				
			\end{enumerate}
		%%% End Solution to Q2
		
		
		%%% Q3
		\item \textbf{Nonnegative Combinations} 
		%%% Begin Solution to Q3
			%%% enumerate 
			\begin{enumerate}
				%%% Solution to Q3 - A
				\item Find a nice representation. \\ 
				\textbf{Solution: } \\ Let $x$ and $y$ be solution to the equation $mx + ny = k$. We know that $x_{nice}$ is bounded to be between $0 \le x < n$. In order words, this means 
							$$x_{nice} = x (\text{mod  } n)$$
				which means that $$n \mid (x_{nice} - x)$$ or in other words, $x_{nice} - x$ is a multiple of $n$. So we can write 
							$$x_{nice} - x = an$$ 
				where $a$ is any integer. Also because $x$ and $y$ are inter-dependent, constraints on $x$ will have some effect on $y$ as well. You have to figure this out yourself. Essentially, we have 
							$$y_{nice} - y = -am$$ 
				
				This implies, 
				$$mx_{nice} + ny_{nice} = m(x + an) + n(y - am) = mx + ny = k$$
				Thus, we know that there exists at least a solution with $x_{nice} = x (\text{mod  } n)$. However, we still have to show that only one such representation is possible. 
				
				For this, assume that there are two solutions $(x, y)$ and $(x', y')$. So we have 
							$$mx + ny = mx' + ny'$$
							$$m(x - x') = n(y' - y)$$
							$$(y' - y) = \frac{m(x - x')}{n}$$
							
				Because $gcd(m, n) = 1$, this implies that in order for $y' - y$ to be an integer, $n \mid (x - x')$ must hold. Because $0 \le x, x' < n$. So $x - x' = 0$, and thus
								$$x = x'$$
				
				Because $x - x' = 0$, the above equation becomes 
							$$m(0) = n(y' - y)$$
							$$y' - y = 0$$
							$$y' = y$$
				Hence, we've shown that there only one nice representation. 
				%%% End Solution to Q3 - A
				
				\item Prove that the largest $k$ that cannot be written as a non-negative integral sum of $m$ and $n$ is $mn - m - n$ \\
				\textbf{Solution: } \\ Following the hint, we will find the maximum possible ways of $x$ and $y$. For $x$ it is pretty evident that
							$$max(x) = n - 1$$
				A non-negative integral sum is a sum where $0 \le x < n$ and $y \ge 0$. However, we are asked to find the largest $k$ that can't be written as a non-negative integral sum. This implies that $y < 0$. Now the question. Given that $y < 0$, whats the maximum possible value of $y$. Obviously,
						$$max(y) = -1$$
				So, we have   
				$$mx + ny = m(n - 1) + n(-1) = mn - m - n$$
				
			\end{enumerate}
		%%% End Solution to Q3
		
		%%% Q4
		\item \textbf{Binary GCD} 
		%%% Begin Solution to Q4
			%%% Enumeration
			\begin{enumerate}
				%%% Begin Q4 - A
				\item Prove that for any positive integers $d$, $x$, and $y$, $d$ divides $gcd(x,y)$ if and only if $d$ divides $x$ and $d$ divides $y$. \\
				%%% Begin Solution to Q4 - A
				\textbf{Solution: } \\ If $d \mid gcd(x, y)$, then
							$$d = ax + by$$
						$$1 = a (\frac{x}{d}) + b (\frac{y}{d})$$
				Because $a, b \in Z$, the above equation holds if and only if $\frac{x}{d}, \frac{y}{d} \in Z$. Therefore, if $d \mid gcd(x, y)$, then $d \mid x$ and $d \mid y$. \cmark
				%%% End Q4 - A
				
				
				%%% Begin Q4 - B
				\item Prove that if $a$ and $b$ are both even, then $gcd(a,b) = 2gcd(a/2,b/2)$. \\
				%%% Solution to Q4 - B
				\textbf{Solution: } \\
				The definition of $gcd(a, b)$ says that $gcd(a, b) = d$ iff $d \mid a$ and $d \mid b$ and $$ax + by = d$$ where $x, y \in Z$. Because both $a$ and $b$ are even, this implies that $d$ is even as well(which implies that $2 \mid a, b, d$). So we have, 
				$$x (\frac{a}{2}) + y (\frac{b}{2}) = (\frac{d}{2})$$
				Note that the above equation is consistent keeping into constraints mentioned in the question($a, b$ are even). So 
					$$gcd(\frac{a}{2}, \frac{b}{2}) = \frac{d}{2}$$
					$$d = 2 \cdot gcd(\frac{a}{2}, \frac{b}{2})$$
				and because $d = gcd(a, b)$
					$$gcd(a, b) = 2 \cdot gcd(\frac{a}{2}, \frac{b}{2}) \text{   \cmark}$$ 
				%%% End Q4 - B
				
				%%% Begin Q4 - C
				\item Prove that if $a$ is odd and $b$ is even, then $gcd(a,b) = gcd(a,b/2)$. \\
				%%% Begin Solution Q4 - C
				\textbf{Solution: } \\
				This proof is quite similar to part(b), except that $a$ is odd here. We already know that if $gcd(a, b) = d$, we have 
								$$ax + by = d$$
				$\forall x, y \in Z$. Because $a$ is odd, and $b$ is even, this means that $d = gcd(a, b)$ has to be odd. So, dividing $b$ by $2$ won't effect the gcd whatsoever. This implies, 
								$$ax + (\frac{b}{2})y = d$$
					$$gcd(a, b) = gcd(a, \frac{b}{2}) \text{   \cmark}$$
				%%% End Q4 - C
				
				
				%%% Begin Q4 - D
				\item Prove that if $a$ and $b$ are both odd, then $gcd(a,b) = gcd((a−b)/2,b)$ where we assume $a \ge b$ \\
				%%% Solution Q4 - D
				\textbf{Solution: } \\
				As $$gcd(a, b) = gcd(a - b, b)$$
				Because both $a$ and $b$ are odd, this means that $a - b$ is even. So we have $gcd(even, odd)$, which resembles a lot like part(c) where, 
						$$gcd(a, b) = gcd(a, \frac{b}{2})$$
				where $a$ was even, but $b$ was odd. So by using part(c), we can write
				$$gcd(a - b, b) = gcd(\frac{a - b}{2}, b)$$
				$$gcd(a, b) = gcd(\frac{a - b}{2}, b) \text{   \cmark}$$
				%%% End Q4 - D
				
				
				%%% Begin Q4 - E
				\item Design an efficient binary gcd algorithm that uses $O(log(max(a,b)))$ subtractions, halving, and parity
				tests.
\begin{lstlisting}
int binary_gcd(int a, int b)
{
	if (a == 0)
	{
		/* return b here */
		return b;
	}	
	/* if a is even */
	if ((a & 1) == 0) /* to check if a is even, because you can't use % */
	{
		/* if b is even */
		if ((b & 1) == 0) /* to check if a is even, because you can't use % */
		{
			/* if both a and b are even, we use part(b) */
			return 2 * binary_gcd(a / 2, b / 2); 
			/* there's a better way than multiply and dividing by 2. Can you figure out what? */
		
		}
		/* if b is not even */
		else
		{
			/* part(c) can help us here */
			return binary_gcd(a / 2, b);
		}
	}
	/* if a is not even */
	else
	{
		/* but b is even */
		if ((b & 1) == 0)
		{
			/* again part(c) can help us here */
			return binary_gcd(b / 2, a);
		}
		/* if both a and b are odd */
		else
		{
			/* part(d) is helpful here */
			if (a > b)
			{
				return binary_gcd(((a - b) / 2), a);
			}
			else
			{
				return binary_gcd(((b - a) / 2), a);
			}
		}
	}
}
\end{lstlisting}
				
		One important that's left to do is to see whether we've meet that constraint $O(log(max(a, b)))$. I'm not explaining it here, because it is a big subject. You should come and talk to me if you want to know why this is the complexity. You might like to see \href{https://en.wikipedia.org/wiki/Binary_GCD_algorithm#Efficiency}{this}, but you will have to go through a lot of detail to get to the conclusion, and thats not described in the wiki page.  
				
				%%% End Q4 - E
				
			\end{enumerate}
		%%% End Solution to Q4
		
		
		%%% Q5
		\item \textbf{Easy RSA} \\
		%%% Solution to Q5
		\textbf{Solution: } \\
		Note that the reason the RSA is considered to be secure is because the factorization of product of two large primes is very difficult. When you decide to use a single prime, you're in trouble. 
		
		Suppose that we wish to send the message $x$. This is plain-text, and we encrypt in using our modified RSA scheme. So, the encrypted message $y$ becomes
					$$y = E(x) \equiv x^e (\text{mod  } p)$$  
		
		So, Eve observes $y = x^e (\text{mod  } p)$ as well as the numbers $p$ and $e$. Now that she wishes to decrypt it, she'd use the RSA decryption function, which is
					$$x = D(y) \equiv x^{ed} (\text{mod  } p)$$
		We already know that $ed \equiv 1 (\text{mo  d} p-1)$. So Eve can follow the following scheme to find the plain-text $x$
		
				\begin{enumerate}
					\item Because $ed \equiv 1 (\text{mod  } p-1)$, $d$ is the multiplicative inverse of $e (\text{mod  } p-1)$. So, find $d$ using the extended euclidean algorithm.  
					\item Compute $y^d (\text{mod  } p)$ by whatever method you know. However, Repeated Squaring is considered to be the most efficient one. 
				\end{enumerate}
		In this way, you'll be able to extract the original contents from the encrypted message. 
		
		%%% End Solution to Q5
		
		%%% Q6
		\item \textbf{RSA} \\
		%%% Solution to Q6
		\textbf{Solution: } \\
		Given $N = pq$ and $\phi(N) = (p - 1)(q - 1)$, we have to solve for $p$ and $q$. Take $q = N / p$, and substitute in the second equation. I leave the algebra up to you, but at the end you should have something which looks like 
		$$ p^2 +  (\phi(N) - N - 1)p  + N  $$
		
		Using the quadratic equation, solve for $p$ and from the value of $p$, solve for $q$ because $q = N / p$. Note that you'll get two solutions for $p$ and $q$. Which is the correct one? Or both are correct? I leave this up to you to decide.  
		%%% End Solution to Q6
		
		
		%%% Q7
		\item \textbf{Modular Arithmetic} 
		%%% Begin Solution to Q7
			
			%%% Begin enumerate
			\begin{enumerate}
				\item Modular Arithmetic 5 \\
				\textbf{Solution: } \\
				You people are required to draw the tables yourself.\\
				\textbf{Additive Inverses: } 
					\begin{enumerate}
						\item Additive Inverse of 0 = 0
						\item Additive Inverse of 1 = 4
 						\item Additive Inverse of 2 = 3
						\item Additive Inverse of 3 = 2 
						\item Additive Inverse of 4 = 1
					\end{enumerate}
				
				\textbf{Multiplicative Inverses: }
					\begin{enumerate}
						\item Multiplicative Inverse of 0 = Doesn't exist
						\item Multiplicative Inverse of 1 = 1
						\item Multiplicative Inverse of 2 = 3
						\item Multiplicative Inverse of 3 = 2 
						\item Multiplicative Inverse of 4 = 4
					\end{enumerate}

				\item Solve the equations. 
				
					\begin{enumerate}
						\item $5x + 23 \equiv 6 \text{  mod } 47$ \\
						\textbf{Solution: }\\
						$5x \equiv -17 \text{ mod } 47$
						Now you have to calculate the multiplicative inverse of $5 \text{ mod } 47$. Use the extended GCD to do this. 
						$EGCD(47, 5) = \{1, -2, 19\}$
						So $19$ is the multiplicative inverse of $5 \text{ mod } 47$. Hence, 
						$$x \equiv 19 \cdot -17 \text{ mod } 47 = 6 \text{ mod } 47$$
						
						\item $9x + 80 \equiv 2 \text{  mod } 81$ \\
						\textbf{Solution: }\\
						Inverse doesn't exist, and so there's no solution to $x$.
						
						\item system of simultaneous equations \\
						\textbf{Solution: }\\
						$$30x + 3(4 + 13x) \equiv 0 \text{ mod } 37$$
						$$69x + 12 \equiv 0 \text{ mod } 37$$
						$$32x \equiv −12 \text{ mod } 37$$
						
						So, $EGCD(37, 32) = \{1, 13, -15 \}$. So $-15$ is the multiplicative inverse of $32 \text{ mod } 37$. This implies, 
						$$x \equiv -15 \cdot -12 \text{ mod } 37 \equiv 32 \text{ mod } 37$$. So $y$ becomes 
						$$y \equiv 4 + 13 \cdot 32 \text{ mod } 37 \equiv 13 \text{ mod } 37$$.
						
					\end{enumerate}
				
				\item $gcd(5688, 2010)$
						$$gcd(5688, 2010) = gcd(2010,1668)$$
						$$= gcd(1668, 342)$$
						$$= gcd(342, 300)$$
						$$= gcd(300, 42)$$
						$$= gcd(42, 6)$$
						$$= gcd(6,0)$$
				So, 
						$$gcd(5688, 2010) = 6$$
					
				\item Same as Q2(b)
				
				
			\end{enumerate}
			
			
			%%% End enumerate
		
		%%% End Solution to Q7
		
	%%% end enumeration for the Questions. 
	\end{enumerate}
	
	
	
	
\end{document}